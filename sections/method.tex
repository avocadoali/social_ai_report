\section{Method}

In this section, we detail the methodology employed to determine the discrete integration model
along with the implementation steps undertaken to validate our approach.

\subsection{Dataset Description} \label{sec:dataset_descrition}
For our experimentation, we utilized the inD, exiD, and rounD datasets provided by the 
\textit{Institut für Kraftfahrzeuge (ika) RWTH Aachen University}.
The datasets offer vehicle trajectories recorded at German intersections, highway exits, and entries, 
and roundabouts, respectively. 
Additionally, the datasets were provided with a tool that allowed us to visualize them for better understanding. 
The datasets include 18 columns, spanning from vehicle identifiers and lifetimes of vehicles to various columns of motion-related data points.
We will primarily focus on the following columns:
$x_{\text{Center}}$, $y_{\text{Center}}$, $x_{\text{Velocity}}$, $y_{\text{Velocity}}$, $x_{\text{Acceleration}}$, 
$y_{\text{Acceleration}}$.

The columns  $x_{\text{Center}}$ and $y_{\text{Center}}$ represent the current position of the object's center in the coordinate coordinate system. Preprocessing can be done such that the columns describe the distance to the origin of the vehicle.

Velocity information is captured through columns $x_{\text{Velocity}}$ and $y_{\text{Velocity}}$ representing velocities 
in the x-axis and y-axis respectively.

Acceleration data are represented in columns $x_{\text{Acceleration}}$ and $y_{\text{Acceleration}}$ which provide 
information on the acceleration in the x-axis and y-axis respectively.

\subsection{Model Selection Process} 
Our model selection process involved a systematic trial and error approach with a total of 8 different models. 
Each model underwent an evaluation process to assess its ability to accurately predict vehicle acceleration across 
various scenarios. 
Ultimately, the following two linear models emerged as the most suitable choice based on their superior performance metrics
\begin{align} 
    a_{dis}(k) &= \bar{c}_1 \bigl( s(k) - s(k+1) - v(k) \bigr) -\bar{c}_2 a(k-1) \\
    a_{vel}(k) &= \bar{c}_3 \bigl( v(k) - v(k+1) \bigr) -\bar{c}_4 a(k-1) 
\end{align}

This linear model is then solved by linear regression. 
After training the model on the acceleration set from our dataset, we can rearrange the formulas
to determine the distance and velocity formulas, similar to the ballistic integration
\begin{align} 
    s(k+1) &= s(k) + v(k) + c_1 a_{dis}(k) + c_2 a(k-1) \\
    v(k+1) &= v(k)        + c_3 a_{vel}(k) + c_4 a(k-1)
\end{align}


The constants can then be determined through these calculations:
\begin{align}
   c_1 &= \frac{1}{\bar{c}_1} \\
   c_2 &= \bar{c}_2 \cdot c_1 \\
   c_3 &= \frac{1}{\bar{c}_3} \\
   c_4 &= \bar{c}_4 \cdot c_3
\end{align}

By rearranging the model as such, we specifically ensure that for both $s(k)$ and $v(k)$ we receive the same 
acceleration, as we are training both models on the same acceleration set.
Thus, we receive a proper integration method for the data set on which we trained the model on, which solves the 
problems of the mismatched acceleration in previous attempts.
As seen in the section \ref{sec:dataset_descrition}, we can see, that the columns for distance, velocity and acceleration are 
split into their x- and y-components.
Thus, our linear models will look like this:

\hfil

\textbf{Distance Model} (Acceleration from distance formula):
{\footnotesize
\begin{align} \label{eq:lin_model_acc_dis}
    \begin{bmatrix}
        a_x(k) \\ 
        a_y(k)       
    \end{bmatrix}_{\text{dis}}
    =
    \begin{bmatrix}
       s_x(k) - s_x(k+1) - v_x(k) & -a_x(k-1) \\ 
       s_y(k) - s_y(k+1) - v_y(k) & -a_y(k-1)   
    \end{bmatrix}
    \begin{bmatrix}
        \overline{c}_1 \\
        \overline{c}_2 \\
   \end{bmatrix}
\end{align}
}

\textbf{Velocity Model} (Acceleration from velocity formula):
\begin{align} \label{eq:lin_model_acc_vel}
    \begin{bmatrix}
        a_x(k) \\ 
        a_y(k) 
    \end{bmatrix}_{\text{vel}}
    =
    \begin{bmatrix}
        v_x(k) - v_x(k+1) & -a_x(k+1)    \\ 
        v_y(k) - v_y(k+1) & -a_y(k+1)    \\
    \end{bmatrix}
    \begin{bmatrix}
        \overline{c}_3 \\
        \overline{c}_4 \\
   \end{bmatrix}
\end{align}

\hfil

Afterward, we will compare the accelerations derived from displacement, denoted as 
$\begin{bmatrix} a_x(k) \\ a_y(k) \end{bmatrix}_{dis}$
, with those derived from velocity, denoted as 
$\begin{bmatrix} a_x(k) \\ a_y(k) \end{bmatrix}_{\text{vel}}$
, to assess the effectiveness of the distance model's acceleration estimation relative to that of the velocity model.


Using the coefficients determined from linear regression from the linear models \eqref{eq:lin_model_acc_dis} and \eqref{eq:lin_model_acc_vel}, we can determine the distance and velocity using the following matrices incorporating the x and y components:

\hfil

\textbf{Distance formula}
{\footnotesize
\begin{align}
\label{eq:distance_matrix}
    \begin{bmatrix}
        s_x(k+1) \\ 
        s_y(k+1) 
    \end{bmatrix}
    =
    \begin{bmatrix}
        a_x(k) & a_x(k-1)    \\ 
        a_y(k) & a_y(k-1)    \\
    \end{bmatrix}
    \begin{bmatrix}
        c_1 \\
        c_2 \\
    \end{bmatrix}
    +
    \begin{bmatrix}
        s_x(k) + v_x(k) \\ 
        s_y(k) + v_y(k) \\
    \end{bmatrix}
\end{align} 
}

\hfil

\textbf{Velocity formula}
\begin{align}
\label{eq:velocity_matrix}
    \begin{bmatrix}
        v_x(k+1) \\ 
        v_y(k+1) 
    \end{bmatrix}
    =
    \begin{bmatrix}
        a_x(k) & a_x(k-1)    \\ 
        a_y(k) & a_y(k-1)    \\
    \end{bmatrix}
    \begin{bmatrix}
        c_3 \\
        c_4 \\
   \end{bmatrix}
    +
    \begin{bmatrix}
        v_x(k) \\
        v_y(k) 
    \end{bmatrix}
\end{align}


Thus, we obtain a discrete integration model with matching accelerations for the distance and velocity formula.

\subsection{Evaluation Metrics and Results}

The evaluation of our linear model's performance relies on several widely-used metrics for linear regression: Mean Squared Error (MSE), Mean Absolute Error (MAE), and R-squared ($R^2$) score.

Mean Squared Error (MSE) measures the average squared difference between the actual and predicted values. It penalizes large errors more heavily than smaller ones, making it sensitive to outliers. A lower MSE indicates better model performance.

Mean Absolute Error (MAE) calculates the average absolute difference between the actual and predicted values and is 
not affected by the scale of the data. Like MSE, lower MAE values indicate better model performance.

R-squared ($R^2$) score quantifies the proportion of the variance in the dependent variable that is predictable 
from the independent variables. 
It ranges from 0 to 1, where a value closer to 1 indicates a better fit of the model to the data. 

