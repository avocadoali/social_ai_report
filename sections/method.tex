\section{Method}

In this section, we detail the methodology employed for motion prediction in dynamic environments, 
along with the implementation steps undertaken to validate our approach.

\subsection{Dataset Description} 
For our experimentation, we utilized the inD, exiD, and rounD datasets.
The datasets offer vehicle trajectories recorded at German intersections, highway exits and entries, and roundabouts, respectively. 
Additionally, a repo (drone-dataset-tool) was provided that allowed us to visualize the dataset for better understanding. 


\subsection{Model Selection Process} 
Our model selection process involved a systematic trial and error approach with a total of 8 different models. 
Each model underwent an evaluation process to assess its ability to accurately predict vehicle acceleration across 
various scenarios. 
Ultimately, the following linear model emerged as the most suitable choice based on its superior performance metrics.

\subsection{Training and Testing Procedure} 
To facilitate model training and evaluation, we employed the common data-splitting techniques provided by 
the sci-kit-learn Python library. 
The $train\_test\_split()$ function was utilized, with a test size of 0.3, to ensure a sufficient amount of data 
for robust evaluation.

\subsection{Evaluation Metrics and Results} 
The performance of our linear model was assessed using common metrics for linear regression, including 
R-squared (r2) and Mean Squared Error (MSE). 
Detailed evaluation results, including mock data for the evaluation metrics, will be provided later in the report, 
offering insights into the model's predictive capabilities.

\subsection{Rearrangement of formulas} 
After computing the resulting linear regression model and its coefficients, we can rearrange the formula 
for the acceleration models. 
After adjusting the coefficients we receive the following formulas for the distance and velocity.


\subsection{Notable Findings and Observations} 
During the evaluation process, several notable findings were observed. 
Most notably, after rearranging the formula for distance and velocity, a constant error was detected. 
However, increasing the size of NumPy arrays appeared to mitigate this error, indicating a potential 
floating-point precision issue.

\subsection{Future Directions} 
Based on our implementation experience, future research and improvement areas include enhancing the 
precision of NumPy arrays by increasing their size and expanding the dataset to encompass a more extensive
range of scenarios for comprehensive model training and testing.

